\documentclass[11pt,a4paper,openany]{book}

\input{C:/Users/serru/Documents/Codes/LaTeX/Environnement_package.tex}
\input{C:/Users/serru/Documents/Codes/LaTeX/Environnements_settings.tex}
\input{C:/Users/serru/Documents/Codes/LaTeX/Environnement_theorems.tex}

\setcounter{tocdepth}{4}
\sloppy

\title{}
\author{Serrurot Gabin\\
BTS SNIR}
\date{\today}

\begin{document}

\sloppy

\vspace*{\stretch{1}}
\begin{minipage}{0.9\linewidth}
\rule{\linewidth}{0.5mm}\\[0.2cm]
\huge\bfseries
\begin{center}
Physique
\end{center}
\rule{\linewidth}{0.5mm}\\[0.2cm]
\maketitle
\end{minipage}
\vspace*{\stretch{1}}

\newpage

\tableofcontents

\newpage

\chapter{Théorèmes généraux sur les circuits}

\section{Introduction}

\section{Loi des nœuds}

\section{Loi des mailles}

\section{Loi d'Ohm}

\section{Association de résistances}

\section{Pont diviseur de tensions}

\section{Puissances et énergie électrique}

\newpage

\chapter{Les grandeurs périodiques: généralités}

\section{Les grandeurs variables}

\section{Les grandeurs périodiques}

\section{Composantes continue et variable}

\section{Mesure des valeurs moyenne et efficace de grandeurs alternatives}

\section{Mesure des valeurs moyenne et efficace de grandeurs non alternatives}

\section{Application}

\section{Propriété énergétique}

%\begin{pyverbatim}
%import math
%def racine(a):
%    return math.sqrt(a)
%print(racine(9))
%\end{pyverbatim}
%\bigskip
%Résultat : 
%\begin{pycode}
%import math
%def racine(a):
%    return math.sqrt(a)
%print(racine(9))
%\end{pycode}
%\begin{pyconsole}
%var = 1 + 1
%var
%\end{pyconsole}

\end{document}