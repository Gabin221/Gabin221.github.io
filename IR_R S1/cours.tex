\documentclass[11pt,a4paper,openany]{book}

\input{C:/Users/serru/Documents/Codes/LaTeX/Environnement_package.tex}
\input{C:/Users/serru/Documents/Codes/LaTeX/Environnements_settings.tex}
\input{C:/Users/serru/Documents/Codes/LaTeX/Environnement_theorems.tex}

\setcounter{tocdepth}{4}
\sloppy

\title{M. Ranguis}
\author{Serrurot Gabin\\
BTS SNIR}
\date{\today}

\begin{document}

\sloppy

\vspace*{\stretch{1}}
\begin{minipage}{0.9\linewidth}
\rule{\linewidth}{0.5mm}\\[0.2cm]
\huge\bfseries
\begin{center}
Informatique et Réseau
\end{center}
\rule{\linewidth}{0.5mm}\\[0.2cm]
\maketitle
\end{minipage}
\vspace*{\stretch{1}}

\newpage

\tableofcontents

\newpage

\chapter{Sous réseaux}

Nous avons un très bon \href{https://www.sebastienadam.be/ipcalculator/}{\textit{convertisseur}} en ligne.\\

Nous avons trois classes principales d'adresse IP:
\begin{itemize}
\item classe A: le premier octet va de 0 à 126, le masque est /8
\item classe B: le premier octet va de 128~\footnote{127 est réservé} à 191, le masque est /16
\item classe C: le premier octet va de 192 à 223, le masque est /24\\
\end{itemize}

Prenons le cas de l'adresse réseau $ 192.168.5.0 $. Puisque c'est un réseau de \textbf{classe C}, alors le masque est un $ /24 $, donc $ 255.255.255.0 $. Dans ce premier cas, nous voulons partager ce réseau en 5 sous-réseaux.\\
Notre réseau ressemble donc à \textcolor{red}{nnnnnnnn.nnnnnnnn.nnnnnnnn.hhhhhhhh}. N'objectif maintenant va être de calculer l'\textbf{adresse} de chaque sous-réseaux, la \textbf{plage d'adresse IP}, l'\textbf{adresse de diffusion} ainsi que le \textbf{nombre maximal d'hôtes}.\\
La première chose à faire est de \textbf{chercher} la \textbf{puissance de 2} qui permet d'être \textbf{supérieur ou égal} à 5:
\begin{equation*}
2^{3} = 8 \geq 5
\end{equation*}

Notre masque a donc changé, il est devenu un /27:
\begin{equation*}
1111~1111.1111~1111.1111~1111.1110~0000 = 255.255.255.224
\end{equation*}

Cette fois-ci, notre réseau ressemble à \textcolor{red}{nnnnnnnn.nnnnnnnn.nnnnnnnn.ssshhhhh}. Tout d'abord, nous pouvons remarquer que puisque le réseau ressemble à ceci, les trois premiers octets seront toujours identiques pour chaque sous-réseau. Le travail va donc se passer sur le dernier octet du réseau.\\

Pour le premier sous-réseau, nous devons d'abord calculer son adresse. Pour se faire, nous commençons par placer tous les bits de la partie hôte à 0. Ensuite, dans la partie sub, puisque nous voulons la première adresse, nous mettons tous les bits de la partie sub à 0, ceci nous donne donc tous les bits du dernier octet à 0 donc nous avons $ 192.168.5.0 $. Pour calculer le premier hôte de ce sous-réseau, nous gardons évidemment \textbf{000} pour la partie \textit{subnet} et nous mettons tous les bits de la partie hôte à 0 en ajoutant 1. Ceci permet d'avoir pour le dernier octet \textcolor{red}{0000~0001}. L'adresse du premier hôte est donc $ 192.168.5.1 $. L'adresse du dernier hôte se passe de la même panière que pour le premier hôte sauf que nous n'allons pas mettre tous les bits de la partie hôte à 0 et ajouter 1 mais le contraire: mettre tous les bits de la partie hôte à 1 et enlever 1. Nous avons donc pour le dernier octet \textcolor{red}{0001~1110} ce qui permet d'obtenir comme adresse $ 192.168.5.30 $. L'adresse de diffusion est $ 192.168.5.31 $. Pour connaître le nombre maximum d'hôtes sur un sous-réseau, nous comptons combien nous avons de bits pour la partie hôte et on les met en puissances de deux. Ici, nous avons 5 bits dans la partie hôte donc nous avons $ 2^{5} - 2 = 30 $ hôtes maximum dans ce sous-réseau.\\

Pour le deuxième sous-réseau, nous plaçons le bit le plus à droite de la partie sub à 1 et les autres bits à 0 tout comme ceux de la partie hôte. Nous avons donc un réseau qui ressemble à ceci:\\
\begin{equation*}
0010~0000 = 32
\end{equation*}

L'adresse du réseau est donc $ 192.168.5.32 $. L'adresse du premier hôte se fait en mettant tous les bits de la partie hôte à 0 et en ajoutant 1 donc nous obtenons l'adresse $ 192.168.5.33 $. L'adresse du dernier hôte se trouve en mettant tous les bits de la partie hôte à 1 et en enlevant 1 donc nous avons l'adresse $ 192.168.5.62 $. L'adresse de diffusion est $ 192.168.5.63 $. Le nombre d'hôtes maximum ne change pas.\\

Nous allons donc regrouper les adresses des 8 sous-réseaux formés dans le tableau ci-dessous et prendre les 5 premiers pour répondre au problème initial:
\begin{table}[!h]
\begin{center}
\begin{tabular}{|c|l|l|l|l|c|} %c/l/r pour centré/à gauche/droite
\hline numéro du sous-réseau & adresse & premier hôte & dernier hôte & broadcast & nombre d'hôtes\\
\hline 1 & $ 192.168.5.0 $ & $ 192.168.5.1 $ & $ 192.168.5.30 $ & $ 192.168.5.31 $ & 30\\
\hline 2 & $ 192.168.5.32 $ & $ 192.168.5.33 $ & $ 192.168.5.62 $ & $ 192.168.5.63 $ & 30\\
\hline 3 & $ 192.168.5.64 $ & $ 192.168.5.65 $ & $ 192.168.5.94 $ & $ 192.168.5.95 $ & 30\\
\hline 4 & $ 192.168.5.96 $ & $ 192.168.5.97 $ & $ 192.168.5.126 $ & $ 192.168.5.127 $ & 30\\
\hline 5 & $ 192.168.5.128 $ & $ 192.168.5.129 $ & $ 192.168.5.158 $ & $ 192.168.5.159 $ & 30\\
\hline 6 & $ 192.168.5.160 $ & $ 192.168.5.161 $ & $ 192.168.5.190 $ & $ 192.168.5.191 $ & 30\\
\hline 7 & $ 192.168.5.192 $ & $ 192.168.5.193 $ & $ 192.168.5.222 $ & $ 192.168.5.223 $ & 30\\
\hline 8 & $ 192.168.5.224 $ & $ 192.168.5.225 $ & $ 192.168.5.254 $ & $ 192.168.5.255 $ & 30\\
\hline
\end{tabular}
\caption{Tableau regroupant les adresses remarquables des 8 sous-réseaux.}
\label{}
\end{center}
\end{table}

Nous pouvons remarquer dans ce tableau la régularité du pas pour passer d'un sous-réseau à l'autre. En effet, nous avons ajouté à chaque fois dans le dernier bit 32 par rapport à la ligne précédente. Autrement dit, pour connaître l'adresse $ n $ donnée en connaissant la première adresse, nous pouvons faire $ (n - 1)*32 $. Par exemple, la $ 5^{eme} $ adresse du premier hôte, nous pouvons faire $ (5 - 1)*32 = 4*32 = 128 $. Puisque l'adresse du premier hôte est $ 192.168.5.1 $, alors nous avons $ 192.168.5.1 + 128 = 192.168.5.129 $ ce qui est bien ce que nous lisons dans le tableau.

\end{document}