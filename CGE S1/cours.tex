\documentclass[11pt,a4paper,openany]{book}

\input{C:/Users/serru/Documents/Codes/LaTeX/Environnement_package.tex}
\input{C:/Users/serru/Documents/Codes/LaTeX/Environnements_settings.tex}
\input{C:/Users/serru/Documents/Codes/LaTeX/Environnement_theorems.tex}

\setcounter{tocdepth}{4}
\sloppy

\title{}
\author{Serrurot Gabin\\
BTS SNIR}
\date{\today}

\begin{document}

\sloppy

\vspace*{\stretch{1}}
\begin{minipage}{0.9\linewidth}
\rule{\linewidth}{0.5mm}\\[0.2cm]
\huge\bfseries
\begin{center}
Cours de CGE
\end{center}
\rule{\linewidth}{0.5mm}\\[0.2cm]
\maketitle
\end{minipage}
\vspace*{\stretch{1}}

\newpage

\tableofcontents

\newpage

Il faudra se faire un comte Linkedin.\\
Avoir un CV en ligne tout le temps.\\
Commencer son CV avec une phrase contenant 5 mots clé.\\
Garder le rapport 1/3 - 2/3 entre la colonne de gauche et la colonne de droite.\\
Pourquoi pas mettre un logo des établissements fréquentés.\\
Sens antichronologique.\\
Pourquoi pas faire une "frise chronologique" pour les expériences.\\
Les logos des compétences.\\
QR code vers les pages des réseaux sociaux professionnels.\\
Développer les compétences plus que l'expérience. Dire le contenu du BTS.\\
Pourquoi pas faire un bandeau au dessus du CV avec des lignes d'un code que j'ai fais.\\
Mettre mon expérience pro à gauche et terminer à droite par mes compétences informatiques.\\
Eviter les loisirs.\\
Pour les formations pourquoi ne pas faire une frise chronologiques qui pourrait éviter de noter les années\\


\textbf{!!! trouver 5 mots clés pour se caractériser d'un point de vue professionnel !!!}

\begin{itemize}
\item capacité d'adaptation
\item rigoureux
\item travail en équipe
\item curieux
\item ponctuel
\end{itemize}

\newpage

\section{Exercice 1}

\begin{itemize}
\item \textbf{Pour la pub de BMW}: l'émetteur est la marque BMW, le destinataire est un homme (car la femme de la pub) pouvant acheter une BMW d'occasion, le code est pictural, il s'agit de l'ambiguïté sexuelle et le référent est la publicité. Il s'agit de créer une complicité masculine avec la marque BMW.
\item \textbf{Pour la pub de mini}: l'émetteur est la marque Mini, le destinataire est une personne plutôt féminine du fait des bijoux, de l'élégance reflétée dans la photo, en plus de cela très urbaine par la taille d'une mini. Le code est pictural, il s'agit de l'humour et le référent est la publicité.
\item \textbf{Pour la pub de Bachelorette}: l'émetteur est abc. Le message est de transmettre les horaires de l'émission, le thème de l'émission. Le code est pictural avec la réification de la personne, la rose. Le code est de l'anglais.  
\item \textbf{Pour l'image du chien}: l'émetteur est une organisation contre la violence animale, le code Pénal et les termes juridiques le rappellent. Le récepteur est chaque propriétaire d'animaux, le grand publique et les personnes assistant à des violences animales. Le code est verbal avec du français.
\end{itemize}

\section{Exercice 2}

Le message visuel est de choquer les personnes regardant l'image. La fonction est informative. L'image est impressive. Convaincre aussi.

\end{document}